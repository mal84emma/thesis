% ******************************************************************************
% ****************************** Custom Margin *********************************

% Add `custommargin' in the document class options to use this section
% Set {innerside margin / outerside margin / topmargin / bottom margin}  and
% other page dimensions
\ifsetCustomMargin
  %\RequirePackage[left=37mm,right=30mm,top=35mm,bottom=30mm]{geometry}
  \usepackage{geometry}
  \geometry{
    a4paper,
    left=25mm,right=20mm,
    top=35mm,bottom=25mm,
    head=20mm,foot=20mm,
    ignorefoot
  }
  \setFancyHdr % To apply fancy header after geometry package is loaded
\fi

% Add spaces between paragraphs
%\setlength{\parskip}{0.5em}
% Ragged bottom avoids extra whitespaces between paragraphs
\raggedbottom
% To remove the excess top spacing for enumeration, list and description
%\usepackage{enumitem}
%\setlist[enumerate,itemize,description]{topsep=0em}

% *****************************************************************************
% ******************* Fonts (like different typewriter fonts etc.)*************

% Add `customfont' in the document class option to use this section

\ifsetCustomFont
  % Set your custom font here and use `customfont' in options. Leave empty to
  % load computer modern font (default LaTeX font).
  %\RequirePackage{helvet}

  % Set a nice friendly font for readability
  \usepackage[T1]{fontenc}
  \usepackage{tgheros} % set font, for alternatives see https://www.overleaf.com/learn/latex/Font_typefaces
  \renewcommand*\familydefault{\sfdefault}

  % For use with XeLaTeX
  %  \setmainfont[
  %    Path              = ./libertine/opentype/,
  %    Extension         = .otf,
  %    UprightFont = LinLibertine_R,
  %    BoldFont = LinLibertine_RZ, % Linux Libertine O Regular Semibold
  %    ItalicFont = LinLibertine_RI,
  %    BoldItalicFont = LinLibertine_RZI, % Linux Libertine O Regular Semibold Italic
  %  ]
  %  {libertine}
  %  % load font from system font
  %  \newfontfamily\libertinesystemfont{Linux Libertine O}
\fi

\usepackage{eurosym} % euro symbol

% *****************************************************************************
% **************************** Custom Packages ********************************

% ************************* Algorithms and Pseudocode **************************

%\usepackage{algpseudocode}

% ************************* Math **************************

\usepackage{amsmath,amssymb,amsfonts} % for subsequations
\usepackage{relsize} % for mathlarger
% symbols
\usepackage{gensymb}
\usepackage{commath}
\usepackage{mathrsfs}
\usepackage{bbm}
\usepackage{pifont}
\usepackage{siunitx}
%\usepackage{array}

% Define even bigger font sizes for eqns
\makeatletter
\newcommand{\vast}{\bBigg@{3}}
\newcommand{\Vast}{\bBigg@{4}}
\makeatother

% spacing in equations
\newcommand{\eqnskip}{1em}
\newcommand{\smalleqnskip}{0ex}

% ************************* Colours **********************************
\usepackage[dvipsnames]{xcolor}
% see https://www.overleaf.com/learn/latex/Using_colors_in_LaTeX#Accessing_additional_named_colors
\definecolor{linkcolor}{HTML}{505050}
\definecolor{citecolor}{HTML}{000075}
\definecolor{urlcolor}{HTML}{CE0058}

% User defined colors
\newcommand{\ph}[1]{{\color{red} #1}}
\newcommand{\gray}[1]{\textcolor{gray}{#1}}
% Colouring command for revision edits
\newcommand{\edit}[1]{\textcolor{black}{#1}}
\newcommand{\edittwo}[1]{\textcolor{black}{#1}}

\usepackage[hooks]{tcolorbox} % docs, https://anorien.csc.warwick.ac.uk/mirrors/CTAN/macros/latex/contrib/tcolorbox/tcolorbox.pdf
\tcbuselibrary{skins,breakable}
\newtcolorbox{cbox}[2][]{ % custom box style
  breakable,
  parbox=false,
  boxrule=0mm,
  boxsep=2mm,
  top=3mm,
  bottom=3mm,
  arc=3mm,
  outer arc=3mm,
  left skip=5mm,
  right skip=5mm,
  title={#2},
  colback=RubineRed!5!white,
  fonttitle=\bfseries,
  #1
}
\newtcolorbox[auto counter, number within=chapter,
number freestyle={\noexpand\thechapter.E.\noexpand\arabic{\tcbcounter}}]{ebox}[2][]{ % custom box style
  breakable,
  parbox=false,
  boxrule=0mm,
  boxsep=2mm,
  top=2mm,
  bottom=2mm,
  left=2mm,
  right=2mm,
  arc=3mm,
  outer arc=3mm,
  left skip=.5mm,
  right skip=.5mm,
  title={\thetcbcounter \: #2},
  colback=black!5!white,
  fonttitle=\bfseries,
  colbacktitle=black!50!white,
  toptitle=1mm,
  #1
}

% ********************Captions and Hyperreferencing / URL **********************

% Captions: This makes captions of figures use a boldfaced small font.
%\RequirePackage[small,bf]{caption}

\RequirePackage[labelsep=space,tableposition=top]{caption}
\renewcommand{\figurename}{Fig.} %to support older versions of captions.sty

\usepackage{cleveref} % Referencing without need to explicitly state fig /table
\usepackage{nameref} % for referencing sections by name
\usepackage{hyperref}
\hypersetup{
  unicode=true,
  colorlinks=true,
  linkcolor=citecolor,
  citecolor=citecolor,
  urlcolor=urlcolor,
  linktoc=page
}
\urlstyle{same} % same styling for url as surrounding text

% fancy underlining
%https://tex.stackexchange.com/questions/36894/underline-omitting-the-descenders
\usepackage{soul}
\makeatletter
\newcommand*{\whiten}[1]{\llap{\textcolor{white}{{\the\SOUL@token}}\hspace{#1pt}}}
\DeclareRobustCommand*\myuline{%
    \def\SOUL@everyspace{\underline{\space}\kern\z@}%
    \def\SOUL@everytoken{%
     \setbox0=\hbox{\the\SOUL@token}%
     \ifdim\dp0>\z@
        \raisebox{\dp0}{\underline{\phantom{\the\SOUL@token}}}%
        \whiten{1}\whiten{0}%
        \whiten{-1}\whiten{-2}%
        \llap{\the\SOUL@token}%
     \else
        \underline{\the\SOUL@token}%
     \fi}%
\SOUL@}

% *************************** Graphics and figures *****************************

%\usepackage{rotating}
%\usepackage{wrapfig}

% Uncomment the following two lines to force Latex to place the figure.
% Use [H] when including graphics. Note 'H' instead of 'h'
%\usepackage{float}
%\restylefloat{figure}

% Subcaption package is also available in the sty folder you can use that by
% uncommenting the following line
% This is for people stuck with older versions of texlive
%\usepackage{sty/caption/subcaption}
\usepackage{subcaption}

% ********************************** Tables ************************************
\usepackage{booktabs} % For professional looking tables
\usepackage{tabularx}
\usepackage{colortbl}
\usepackage{multicol}
\usepackage{multirow}
\usepackage{makecell}
\usepackage[skip=10pt]{caption} % increase space between caption and table

%\usepackage{longtable}

% centered adaptive column for tabularx
\newcolumntype{Y}{>{\centering\arraybackslash}X}

% ********************************** Tikz & plotting ************************************
\usepackage{tikz}
\usetikzlibrary{arrows.meta}
\usetikzlibrary{calc}
\usetikzlibrary{shapes}
\usetikzlibrary{fit,backgrounds}
\usetikzlibrary{decorations.pathreplacing,calligraphy}
\usetikzlibrary{patterns}

\usepackage{pgfplots}
\usepgfplotslibrary{fillbetween}
% define gaussian pdf and cdf
\pgfmathdeclarefunction{sig}{2}{%
  \pgfmathparse{1/(1+exp(#2-#1))}%
}

% *********************************** SI Units *********************************
\usepackage{siunitx} % use this package module for SI units


% ******************************* Line Spacing *********************************

% Choose linespacing as appropriate. Default is one-half line spacing as per the
% University guidelines

% \doublespacing
\onehalfspacing
% \singlespacing

% ************************ Formatting / Footnote *******************************

% Don't break enumeration (etc.) across pages in an ugly manner (default 10000)
%\clubpenalty=500
%\widowpenalty=500

\usepackage[bottom,hang,flushmargin]{footmisc} % Range of footnote options
\counterwithout{footnote}{chapter} % continuous footnote numbering across chapters

\newcommand\fnsep{\textsuperscript{,}} % footnote separator

% fn to refer back to footnotes
\makeatletter
\newcommand\footnoteref[1]{\protected@xdef\@thefnmark{\ref{#1}}\@footnotemark}
\makeatother

% reduce footnote font size: font size, line size
\renewcommand{\footnotesize}{\fontsize{9pt}{10pt}\selectfont}

% add space between footnote number and text
\let\oldfootnote\footnote
\renewcommand\footnote[1]{%
\oldfootnote{\hspace{1mm}#1}}

% Add support for custom footnote markers
\newcommand{\customfootnotetext}[2]{{% Group to localize change to footnote
  \renewcommand{\thefootnote}{#1}% Update footnote counter representation
  \footnotetext[0]{#2}}}% Print footnote text


% ************************ Glossaries *******************************
\usepackage[acronym,toc,automake,nogroupskip]{glossaries-extra} % package for making glossaries
\makeglossaries
% usage is explained here, https://gb.mirrors.cicku.me/ctan/macros/latex/contrib/glossaries-extra/glossaries-extra-manual.pdf
% commands are: \gls{}, \glsxtrfull{}, \glsxtrlong{}, or \glsxtrshort{}

% align glossary long entries
\setglossarystyle{long}
\renewcommand{\glsnamefont}[1]{\textbf{#1}}

% load in abbreviation definitions
% ******************************************************************************
% ****************************** List of Abbreviations *********************************

% \newacronym{⟨label⟩}{⟨abbrv⟩}{⟨full⟩}

\newacronym{pv}{PV}{Photovoltaic}
\newacronym{caes}{CAES}{Compressed air energy storage}
\newacronym{tes}{TES}{Thermal energy storage}
\newacronym{vrfb}{VRFB}{Vanadium Redox Flow battery}
\newacronym{nas}{NaS}{Sodium--Sulfur High Temperature battery}
\newacronym{li-ion}{Li-ion}{Lithium ion battery}

\newacronym{ev}{EV}{Electric Vehicle}
\newacronym{bev}{BEV}{Battery Electric Vehicle}

\newacronym{ac}{AC}{Air-Conditioning}
\newacronym{hvac}{HVAC}{Heating, Ventilation, and Air Conditioning}
\newacronym{ashp}{ASHP}{Air Source Heat Pump}
\newacronym{gshp}{GSHP}{Ground Source Heat Pump}
\newacronym{cop}{COP}{Coefficient of Performance}
\newacronym{des}{DES}{Distributed Energy System}
\newacronym{nzeb}{NZEB}{Net Zero Energy Building}

\newacronym{tmy}{TMY}{Typical Meteorological Year}

\newacronym{soc}{SoC}{State of Charge}
\newacronym{fos}{FoS}{Factor of Safety}

\newacronym{duos}{DUoS}{Distributed Use of Service}

\newacronym{lcoe}{LCOE}{Levelised Cost of Energy}
\newacronym{lcox}{LCOX}{Levelized Cost of eXergy}
\newacronym{npv}{NPV}{Net Present Value}

\newacronym{ccus}{CCUS}{Carbon Capture, Utilization, and Storage}


\newacronym{fpca}{fPCA}{Principal Component Analysis}

\newacronym{sp}{SP}{Stochastic Programming}
\newacronym{lp}{LP}{Linear Programming}
\newacronym{ro}{RO}{Robust Optimisation}
\newacronym{milp}{MILP}{Mixed Integer Linear Programming}
\newacronym{mpc}{MPC}{Model Predictive Control}
\newacronym{rbc}{RBC}{Rule Based Control}
\newacronym{rl}{RL}{Reinforcement Learning}

\newacronym{ua}{UA}{Uncertainty Analysis}
\newacronym{sa}{SA}{Sensitivity Analysis}
\newacronym{mc}{MC}{Monte Carlo}
\newacronym{gsa}{GSA}{Global Sensitivity Analysis}
\newacronym{mga}{MGA}{Modelling to Generate Alternatives}

\ifdefineSpeech % if speech mode is defined, use capitalised acronyms for better pronunciation
 \newacronym{voi}{VOI}{Value of Information}
 \newacronym{voo}{VOO}{Value of Optionality}
\else
 \newacronym{voi}{VoI}{Value of Information}
 \newacronym{voo}{VoO}{Value of Optionality}
\fi

\newacronym{evpi}{EVPI}{Expected Value of Perfect Information}
\newacronym{evppi}{EVPPI}{Expected Value of Partial Perfect Information}
\newacronym{evii}{EVII}{Expected Value of Imperfect Information}
\newacronym{evsi}{EVSI}{Expected Value of Sample Information}
\newacronym{evo}{EVO}{Expected Value of Optionality}

\newacronym{evp}{EVP}{Expected Value Problem}
\newacronym{vss}{VSS}{Value of Stochastic Solution}
\newacronym{cvar}{CVaR}{Conditional Value-at-Risk}
%******************************************************************************
%\glsaddall

% style glossary
\setlength\LTleft{0pt}
\setlength\LTright{0pt}
\setlength\glsdescwidth{0.8\hsize}


% ************************ Titling *******************************
\usepackage[explicit]{titlesec}

% *****************************************************************************
% *************************** Bibliography  and References ********************

%\usepackage{cleveref} %Referencing without need to explicitly state fig /table

% Add `custombib' in the document class option to use this section
\ifuseCustomBib
  % \RequirePackage[round, sort&compress, authoryear]{natbib} % CustomBib

  % % make citations a single link, https://tex.stackexchange.com/a/27235
  % \makeatletter
  % \renewcommand\hyper@natlinkbreak[2]{#1}
  % \makeatother

  % If you would like to use biblatex for your reference management, as opposed to the default `natbibpackage` pass the option `custombib` in the document class. Comment out the previous line to make sure you don't load the natbib package. Uncomment the following lines and specify the location of references.bib file

  \RequirePackage[%
    backend=biber,
    style=authoryear-icomp,
    citestyle=authoryear-comp,
    minnames=1,
    maxnames=5,
    mincitenames=1,
    maxcitenames=2,
    minbibnames=3,
    sorting=nty,
    natbib=true%
    ]{biblatex}
  \addbibresource{References/references.bib} % Location of references.bib only for biblatex, Do not omit the .bib extension from the filename.
  % for options see:
  % https://mirrors.ibiblio.org/CTAN/macros/latex/contrib/biblatex/doc/biblatex.pdf
  % (styles) https://www.overleaf.com/learn/latex/Biblatex_bibliography_styles


  % command to allow use of fullcite with different maxnames
  \newcommand{\printpublication}[1]{\AtNextCite{\defcounter{maxnames}{99}}\fullcite{#1}}

  % remove url visited dates
  \AtEveryBibitem{% for bibliography
    \clearfield{urlyear}%
    \clearfield{urlmonth}%
  }
  \AtEveryCitekey{% for fullcite
    \clearfield{urlyear}%
    \clearfield{urlmonth}%
  }

  % make author name bold, solution from https://tex.stackexchange.com/a/327046
  \newcommand*{\boldname}[3]{%
    \def\lastname{#1}%
    \def\firstname{#2}%
    \def\firstinit{#3}}
  \boldname{}{}{}

  \renewcommand{\mkbibnamegiven}[1]{%
    \ifboolexpr{ ( test {\ifdefequal{\firstname}{\namepartgiven}} or test {\ifdefequal{\firstinit}{\namepartgiven}} ) and test {\ifdefequal{\lastname}{\namepartfamily}} }
    {\mkbibbold{#1}}{#1}%
  }

  \renewcommand{\mkbibnamefamily}[1]{%
    \ifboolexpr{ ( test {\ifdefequal{\firstname}{\namepartgiven}} or test {\ifdefequal{\firstinit}{\namepartgiven}} ) and test {\ifdefequal{\lastname}{\namepartfamily}} }
    {\mkbibbold{#1}}{#1}%
  }

  \boldname{Langtry}{Max}{M.}

\fi

% changes the default name `Bibliography` -> `References'
\renewcommand{\bibname}{References}


% *****************************************************************************
% *************************** Position dependent loads ************************

\usepackage{xurl} % load after biblatex to allow url breaks in bibliography
\usepackage{lipsum}


% ******************************************************************************
% ************************* User Defined Commands ******************************
% ******************************************************************************

% *********** To change the name of Table of Contents / LOF and LOT ************

%\renewcommand{\contentsname}{My Table of Contents}
%\renewcommand{\listfigurename}{My List of Figures}
%\renewcommand{\listtablename}{My List of Tables}


% ********************** TOC depth and numbering depth *************************

\setcounter{secnumdepth}{2}
\setcounter{tocdepth}{2}


% ******************************* Nomenclature *********************************

% To change the name of the Nomenclature section, uncomment the following line

%\renewcommand{\nomname}{Symbols}


% ********************************* Appendix ***********************************

% The default value of both \appendixtocname and \appendixpagename is `Appendices'. These names can all be changed via:

%\renewcommand{\appendixtocname}{List of appendices}
%\renewcommand{\appendixname}{Appndx}

% *********************** Configure Draft Mode **********************************

% Uncomment to disable figures in `draft'
%\setkeys{Gin}{draft=true}  % set draft to false to enable figures in `draft'

% These options are active only during the draft mode
% Default text is "Draft"
%\SetDraftText{DRAFT}

% Default Watermark location is top. Location (top/bottom)
%\SetDraftWMPosition{bottom}

% Draft Version - default is v1.0
%\SetDraftVersion{v1.1}

% Draft Text grayscale value (should be between 0-black and 1-white)
% Default value is 0.75
%\SetDraftGrayScale{0.8}


% ******************************** Fonts **********************************
\makeatletter
\newcommand\HUGE{\@setfontsize\Huge{40}{45}}
\makeatother

\makeatletter
\newcommand\VERYHUGE{\@setfontsize\Huge{95}{100}}
\makeatother

\usepackage{fontawesome}


% ********************************* Bullet point styles *********************************
\renewcommand\labelitemi{\raisebox{0.25ex}{$\bullet$}}
\renewcommand\labelitemii{\raisebox{0.2ex}{\textbf{--}}}
\renewcommand\labelitemiii{\raisebox{0.25ex}{\small$\bullet$}}


% ******************************** Todo Notes **********************************
%% Uncomment the following lines to have todonotes.

%\ifsetDraft
%	\usepackage[colorinlistoftodos]{todonotes}
%	\newcommand{\mynote}[1]{\todo[author=kks32,size=\small,inline,color=green!40]{#1}}
%\else
%	\newcommand{\mynote}[1]{}
%	\newcommand{\listoftodos}{}
%\fi

% Example todo: \mynote{Hey! I have a note}

% ******************************** Highlighting Changes **********************************
%% Uncomment the following lines to be able to highlight text/modifications.
%\ifsetDraft
%  \usepackage{color, soul}
%  \newcommand{\hlc}[2][yellow]{{\sethlcolor{#1} \hl{#2}}}
%  \newcommand{\hlfix}[2]{\texthl{#1}\todo{#2}}
%\else
%  \newcommand{\hlc}[2]{}
%  \newcommand{\hlfix}[2]{}
%\fi

% Example highlight 1: \hlc{Text to be highlighted}
% Example highlight 2: \hlc[green]{Text to be highlighted in green colour}
% Example highlight 3: \hlfix{Original Text}{Fixed Text}

% *****************************************************************************
% ******************* Better enumeration my MB*************
\usepackage{enumitem}
