%!TEX root = ../thesis.tex
%*******************************************************************************
%*********************************** Conclusions *****************************
%*******************************************************************************

\chapter{Conclusions} \label{chap:conclusion}

\begin{cbox}[colback=black!5!white]{}
    \large
    We can only make good decisions with good data.\\
    So we need to understand how the data we have affects our decision making, and what good data looks like.
\end{cbox}

\hfill \\

\noindent
% start with context and overarching research question/aim of thesis - c. 1 para
This thesis explores the role of data for supporting decision making in energy systems. Its main aim was to quantify the improvement in decision making that data collection provides for the design and operation energy systems. This was then used to determine whether collecting data to inform these decisions is worthwhile, and if so how much data should be collected, as well as how the decision making process should be adapted to best use this data and the uncertainty reduction it provides.
The computational experiments in this thesis investigate these questions of data collection requirements and the importance of uncertainty reduction for decision making across a broad range of decision making tasks, spanning design, operation, and management, and covering energy system scales from individual buildings to grid-scale energy parks.

\newpage
% then briefly summarise the findings and importance of each chapter, include the lit review, and try to weave some narrative & connections through them - c. 2-3 pages
A literature review demonstrated that while modelling and accounting for uncertainties during decision making is common practice in the energy systems field, little attention has been paid to how reducing these uncertainties can improve decision making. Similarly, while the importance of collecting data from energy systems is fully accepted, and its practical issues widely acknowledged, few studies have looked at the data collection requirements of energy systems, and many costs associated with data collection are frequently ignored. No existing study has quantified the benefit that uncertainty reduction from practical data collection would provide to decision making in energy systems, and determined whether this benefit is greater than the cost of collecting the data, i.e. whether the data collection is worthwhile.

\Cref{chap:methodology} demonstrated how \glsxtrlong{voi} analysis (\glsxtrshort{voi}) can be used to address this research gap, as it provides a numerical framework for quantifying the improvement in decision making provided by uncertainty reduction. The theory and operation of the \glsxtrshort{voi} calculations were explained, and alongside this a simple example problem involving the sizing of solar panels and battery storage in a commercial building was developed to illustrate how \glsxtrshort{voi} can be applied to decision making in energy systems.

The capabilities of \glsxtrshort{voi} to inform decision making about data collection were further demonstrated in \Cref{chap:demonstrations}. It explored three example decision problems in building energy systems: air source heat pump maintenance scheduling, ventilation scheduling for indoor air quality, and ground-source heat pump system design. In each example a different aspect of how \glsxtrshort{voi} can be used to support decision making was illustrated. Across the three examples, there was a large variation in the benefit that each of the data collection options considered provided to decision making. Knowing the \glsxtrshort{ashp} performance and degradation reduced its operation cost by just 0.06\% on average through improved maintenance scheduling. Whereas knowing the office occupancy and improving ventilation scheduling reduced operating costs by 15\% on average. While knowing the ground thermal conductivity to help select the best borehole length for a \glsxtrshort{gshp} system reduced the lifetime cost by around 4\% on average, and in this case a ground thermal test with a moderate precision was found to be most economical. This variation in \glsxtrshort{voi} results highlights the importance of performing \glsxtrshort{voi} analysis, as the value of data collection for supporting decision making is often not obvious even to those with expert knowledge of the energy system.

However, it's not always possible to use \glsxtrshort{voi} when investigating the benefits of data collection and data requirements. \Cref{chap:forecasting} studied the impact of data on the accuracy of machine learning based forecasting models for building operational conditions, and the resulting operational performance when they are used in a \glsxtrlong{mpc} scheme to control distributed generation and storage in a district energy system.
Due to the complexity of the historical building load data, for example the changing energy usage patterns over the dataset, it was not possible to construct a probabilistic model of the data, and so \glsxtrshort{voi} couldn't be used. Instead, the historic dataset of building electricity metering data from the Cambridge University estate was used as a case study, and the analysis considered how the forecasting model accuracies would change if more of the data were available when they were trained.
A simple linear prediction model was found to achieve similar forecast accuracy to high-complexity, state-of-the-art machine learning models, but was more data efficient and provided better generalisation performance. This was partially due to better computational efficiency of the simple model which meant that it could use a larger context window within the computational budget, and so use more information to support its predictions. These results indicate that in this case, the data used for decision making was more important than the sophistication of the model.
Using more than 2 years of hourly resolved load data for model training did not significantly improve the forecasting accuracy, but using less than 1 year led to substantially worse predictions. This indicates that between 1 and 2 years of load data is needed to train an accurate prediction model for building load. However the data used for training is important. Screening training data using change-point analysis to remove low similarity data was able to simultaneously improve data efficiency and prediction accuracy, provided at least 1 year of training data was kept. But it was also found that this data does not necessarily need to come from the target building. Reusing load prediction models trained on data from similar buildings, selected using a data-driven similarity metric, led to prediction errors only 11\% higher than models trained on data from the target building. Therefore, the cost of collecting building specific data for model training should be weighed against the benefits that 11\% better prediction accuracy provides for building control. To investigate this, the relationship between forecast accuracy and operational performance of \glsxtrshort{mpc} was quantified using synthetic forecasts. The \glsxtrshort{mpc} scheme was much more sensitive to the prediction accuracy of grid electricity price and carbon intensity than building load, suggesting that it is more important to focus expenditure there, for example by purchasing commercial forecasts.

\Cref{chap:districts} considered the building load data requirements for the design of a similar district energy system. It quantified the economic benefit of using hourly building energy usage data to support the sizing
of solar-battery systems to decarbonise a district of buildings under grid constraints. Probabilistic models of building load profiles for the case study district were constructed using the same dataset of historic metering data from the Cambridge University estate.
The uncertainty in building load was found to have a significant impact on the operating cost, causing a $\pm$30\% variation in total cost for the energy system designed without monitoring data available. Also, reducing uncertainty in the building load by collecting monitoring data had a significant effect on the optimized energy system design, with total installed capacities of solar, battery, and grid connection varying by roughly $\pm$20\% across the hypothesised measurement values. Therefore, existing methods which stop at either of these two points would conclude that collecting hourly metering data would have a significant effect on system design.
However, applying the On-Policy \glsxtrshort{voi} framework showed that reducing load uncertainty via monitoring would reduce the overall energy system cost by less than 1.5\% on average through improved design. This was less than the estimated cost of getting the load data at the time of design, and so using hourly load monitoring data to support the design of the district energy system was found to be not economically worthwhile.
Additionally, reducing uncertainty in only the mean load was found to provide most of the available decision support benefit. This means that for building retrofits where the mean building load can be well estimated from historic monthly energy usage data, using hourly metering data would provide little additional benefit for system design, and so should only be used if easily available.
With the results from \Cref{chap:forecasting}, this indicates that hourly load data from a building is not needed before a solar-battery system can be designed and installed. As the distribution of load profiles from other buildings can be used to make a sufficiently good system design, and a load prediction model can be trained on data from another, similar building to make a sufficiently good control scheme.
This is important for building energy system designers and operators, as it allows them to avoid wasteful delays to monitor building loads before installation.

In contrast, \Cref{chap:parks} found that reducing uncertainty in the performance of energy storage technologies had a significant impact on energy park design. It investigated the impact of uncertainty in the cost, lifetime, and efficiency of energy storage technologies on the sizing of assets in an energy park, and the benefit of retaining optionality in storage technology choice through the design process. It quantified the improvement in design (reduction in average total cost) achieved when multiple technologies are developed to retain optionality in technology choice, and allow a better storage selection to be made after uncertainty in performance has been reduced. Four candidate storage technologies were considered for use in the park: \glsxtrshort{li-ion}, \glsxtrshort{nas}, \glsxtrshort{vrfb}, and \glsxtrshort{caes}.
When only a single storage technology was selected for development, updating the system design (wind, solar, and storage sizings) after R\&D, when improved information on storage performance is available (i.e. with reduced uncertainty), was found to reduce total costs by 18\% on average. This showed that flexibility in procurement contracts for generation and storage is highly valuable to energy park developers.
If all four energy storage technologies were developed, so that when the system design was updated any technology could be selected for the final design, the average total cost was found to reduce by 13\% compared to the case without optionality. This cost saving was substantially greater than the cost performing R\&D for the three additional technologies. This means retaining optionality in storage technology choice is worthwhile, and so developers should hedge their bets against their choice of storage technology and provide themselves the option to choose the best option once they have better information. However, if two storage technologies were used in the energy park, the average total system cost was 14\% lower than the base case without optionality, and providing storage technology optionality was no longer worthwhile.
These results were shown to be robust to the level of uncertainty reduction in energy storage performance provided by R\&D used in the statistical model, as well as the grid electricity price and carbon emissions data used, and the cost of developing demonstrator storage systems assumed.
So for energy park developers, uncertainty in energy storage technology performance is really important for their decision making. Having the option to adjust their design decisions after better information about storage performance is available can significantly reduce costs, in this case by 31\%.\\

% try to finish with a unifying conclusion paragraph to bring the pieces together - c. 1/2 page
The results of this thesis demonstrate the importance of understanding the impact that data has on decision making. By quantifying the effect that data collection and uncertainty reduction have on their decisions, decision makers can either achieve much better outcomes by using information effectively at the right stage in their decision making process, or know when they can make sufficiently good decisions with less data and so avoid wasting resources collecting data that provides little benefit. In the context of decarbonising energy systems, this understanding can unlock substantial cost and carbon savings purely by adjusting the way we make decisions and the data we use to support them. It can also help us identify what we need to know about our energy systems to design and operate them effectively, and determine when we can make good decisions here and now without wasting time waiting for data. The examples in the thesis show that these opportunities exist across all decision making areas in energy systems, from design to operation to management, and across all scales of energy system, from individual buildings to grid-scale energy parks.

The thesis also demonstrated that \glsxtrlong{voi} analysis provides a clear and rigorous methodology for quantifying the benefit that data collection and uncertainty reduction provide to decision making, and showed how it can be used to justify, rationalise, and prioritise expenditure on data collection in energy systems. The methodology is versatile, and can be used to determine whether data collection is worthwhile to support decision making, which and how much data is most economical to collect, and whether it is worthwhile retaining decision optionality until better information is available. It can do so in complex decision making contexts, such as optimisation based design and control. As the value of data collection is often not obvious even to those with expert knowledge of energy systems, it is important to use this systematic approach to understand how data can support decision making in each energy system context.

\newpage
\section{Limitations and further work}

% discuss limitations of work, and suggest further work - c. 1-1.5 pages
This thesis presented a series of case studies investigating the impact of particular data collection options on specific decision making tasks in given energy systems. While these case studies aimed to cover a broad range of collected data, decision making contexts, and energy system scales, they are necessarily limited and cannot cover all variations and possibilities that are of interest to energy system decision makers.

The key limitation of this work is the lack of generalisation guarantees for the recommendations about how data should be collected and used to support decision making provided by the numerical results. \glsxtrshort{voi} values are only valid under the modelling assumptions used in the calculations, and even innocuous looking changes to the model can lead to very different \glsxtrshort{voi} results and conclusions\footnote{This is discussed in more detail in \Cref{sec:methodology-limitations}.}.
Similarly, the recommendations about how data should be used to train load forecasting models from \Cref{chap:forecasting} are only evidenced for the university buildings from the Cambridge Estates dataset (see Appendix \ref{app:data}). There is no guarantee that this aspect of the dataset is representative of buildings from other universities, for example those in different climates or those with different envelopes and usage patterns, or for other types of buildings, such as residential or commercial buildings.
Further, from results on the value of one type of data for supporting a particular decision, we cannot make conclusions about whether other data might be worthwhile collecting, or whether the same data might be worthwhile for supporting a different decision. For example, while \Cref{chap:districts} showed that collecting hourly load monitoring data was not worthwhile for informing the sizing of solar-battery systems, we cannot know from this whether improved information about the cost of the solar panels or batteries would have a significant impact on the design decision\footnote{These uncertainties were not included in the decision model used in \Cref{chap:districts}.}, or whether collecting hourly load data would be worthwhile to support the sizing of the heat pumps in the buildings.\\

Significant further research is required to determine if the conclusions from this thesis hold across different energy system contexts, such as location, usage type, and scale. Future work should also use the methodologies outlined in this thesis to investigate other decision tasks and data collection options across the energy systems field, to identify opportunities to improve decision making by providing the right data at the right time, and to avoid wasting time and resources collecting data that provides little insight.

There are also many more aspects of data collection and usage that could be explored. For instance, investigation of how much data should be collected to support decision making. This would require more sophisticated probabilistic models of uncertainty reduction from repeated measurements, but is otherwise straightforward to use within the \glsxtrshort{voi} framework. \Cref{chap:forecasting} considered how data availability at the time of model training affects forecasting accuracy. This is a static view of data, but as building energy systems are operated, operational data is available to collect in real-time. Which data should be collected and how this data can be best used to improve building control is an important research question to address. \citep{raisch2025AdaptingChangeComparison} has begun to explore the latter question in the context of continual learning for building thermal dynamics.

% Is this paragraph necessary? It's a little tangential to the main point
One of the core ideas of this work is that the purpose of data is to support decisions, and so the value of data should be measured in terms of its ability to improve decision making. \Cref{chap:forecasting} explores how data can be used to develop high accuracy forecasting models, the goal of which is to improve the control of building energy systems, specifically \glsxtrshort{mpc} in the chapter's case study. The final experiment in the chapter investigated the relationship between forecast accuracy and operation performance of the controller, however as it used relatively simple synthetic forecasts the insight into the control performance with practical forecasts is somewhat limited.
Currently in the building \glsxtrshort{mpc} literature, forecast accuracy and control optimisation are investigated separately. However, to properly understand the data collection requirements of practical building control schemes, these two stages need to be considered together. \Cref{chap:forecasting} provides a starting point for this, but further work is needed to understand where prediction accuracy is important for achieving effective control, and the trade-off between the benefits of accuracy and the costs of achieving it.\\

From a technical perspective, there are significant research opportunities for improving the computational efficiency of \glsxtrshort{voi} calculations. For example, surrogate modelling, importance sampling, or multi-fidelity modelling methods could be used to reduce the number of model evaluations needed to get good estimates of \glsxtrshort{voi} values. Identifying appropriate trade-offs between computational cost and \glsxtrshort{voi} estimate accuracy would also require better understanding of the accuracy/error of those \glsxtrshort{voi} estimates\footnote{This is discussed further in \Cref{app:districts-numerics}.}.\\

Overall, this thesis has demonstrated for a small number of cases the importance of understanding how data and uncertainty reduction affect the decisions we make within energy systems, and has shown how analysing this can improve the decisions we make about data collection. There are many more decisions across energy systems than could be improved by thinking about uncertainty in this way.