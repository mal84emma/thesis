% ************************** Thesis Abstract *****************************
% Use `abstract' as an option in the document class to print only the titlepage and the abstract.
\begin{abstract} \label{chap:abstract}

\customfootnotetext{}{{\normalsize\faVolumeUp} {\small\, You can listen to this abstract \href{https://mal84emma.github.io/thesis/abstract.mp3}{here}.}}

Energy systems need to be designed, operated, and managed effectively to make decarbonisation affordable. But there are many aspects of energy systems, such as their cost and the operating conditions they will face, that are not known precisely at the time these decisions need to be made. To manage these uncertainties, decision makers have to hedge their choices against the possible outcomes, missing out on the best possible choice for the actual system state. Collecting data can reduce uncertainty, and enable better decisions. But data is itself costly. This raises two crucial questions, ``How much does collecting data to reduce uncertainty improve decision making? And are the benefits the data provides to decision making worth its cost?''.

Understanding of how data collection and uncertainty reduction affect decision making in energy systems is needed to determine where data should be used to improve decisions, and where wasting time and resources on low insight data can be avoided. However, no existing studies have numerically determined the data requirements for supporting the design and operation of energy systems, or quantified the benefit of practical data collection for improving decision making in energy systems.

This thesis investigates the impact that data and uncertainty reduction have on the design and operation of energy systems across scales. It aims to demonstrate the importance of considering how data affects decision making, and show that \glsxtrlong{voi} analysis (\glsxtrshort{voi}) provides a clear and rigorous methodology for studying this.
It begins by explaining the \glsxtrshort{voi} framework, and then extends it to allow the study of complex decision making problems where decision policies are required, and the benefit of retaining optionality as uncertainty reduces. How \glsxtrshort{voi} can be used to study varying energy system contexts, and the insights it can provide into the role of data collection for supporting decision making, are then demonstrated via three example decision problems.

Next, the effect of training data on the accuracy of machine learning based forecasting models, and the resulting performance of \glsxtrlong{mpc} in a multi-building energy system is analysed. A simple neural model is found to provide equivalent forecast accuracy to state-of-the-art models, but with better data efficiency and generalisability. Further, using more than 2 years of load data for training provides no significant improvement in forecast accuracy, and the accuracy and data efficiency of the models can be simultaneously improved by using change-point analysis to remove non-representative data from the training set. Reusing models achieves prediction accuracies within 10\% of the baseline without using any data from the target building, suggesting good forecasts can be made without load data from a building.

The value of using load monitoring data to support the design of a grid constrained district energy system is then quantified. Uncertainty in building load is found to significantly impact both system operating costs (±30\%) and the optimal system design (±20\%). However, using building monitoring data to improve the sizing of solar-battery systems in the district reduces overall costs by less than 1.5\% on average. This saving is less than the cost of obtaining the load data, and so using monitoring is shown to be not economically worthwhile. This provides the first numerical evidence to support the sufficiency of using standard building load profiles for energy system design, and suggests that good decisions can be made about the design and control of district energy systems without gathering building specific load data.

Finally, the value of having the option to adjust asset sizings for the design of a grid-scale energy park, and change energy storage technology choice, after uncertainty in storage performance has reduced is quantified. In contrast to building load, reducing uncertainty in storage technology performance by building a demonstrator system significantly reduces the cost of the energy park. Updating asset sizings after storage uncertainty is reduced is found to reduce total costs by 18\% on average. While having the option to switch storage technology choice as well reduces costs by a further 13\%, which is substantially greater than the cost of providing storage optionality.

The benefit that data and uncertainty reduction provide to decision making in energy systems is not obvious, even to those with expert knowledge of the systems. The results of this thesis show the importance of systematically studying how data can support decision making, and the limitations of the current static view of data and uncertainty in the energy systems field.

\end{abstract}