\pgfplotsset{
  layers/axis lines on top/.define layer set={
    axis background,
    axis grid,
    axis ticks,
    axis tick labels,
    pre main,
    main,
    axis lines,
    axis descriptions,
    axis foreground,
  }{/pgfplots/layers/standard},
} % from https://tex.stackexchange.com/questions/240359/tikz-axis-border-on-top-grids-below

\begin{figure}[h]
    \centering
    \vspace*{0.25cm}
    \resizebox{0.8\textwidth}{!}{
    \begin{tikzpicture}
        \def\N{100}
        \def\C{5}
        \def\xmax{10}
        \def\ymin{-0.1}
        \def\ymax{1.2}
        \def\xopt{2.44102} % from desmos
        \def\yopt{0.92817} % from desmos
        \def\xint{5.46579}

        \begin{axis}[
            set layers=axis lines on top,
            every axis plot post/.append style={
                mark=none,domain={0}:{10},samples=\N,smooth},
                xmin={-1}, xmax=\xmax,
                ymin=\ymin, ymax={\ymax},
                axis lines=middle,
                axis line style=thick,
                enlargelimits=upper, % extend the axes a bit to the right and top
                %ticks=none,
                xlabel={\large $\sigma^2$},
                ylabel={\glsxtrshort{evii}},
                xtick={{10}},
                xticklabels={{$\sigma^2_{\text{prior}}$}},
                ytick={{1},{0.75}},
                yticklabels={{\normalfont \glsxtrshort{evpi}},{\large $c(e)$}},
                every major tick/.append style={thick, major tick length=7pt, black!50},
                every axis x label/.style={at={(current axis.right of origin)},anchor=west},
                every axis y label/.style={at={(current axis.above origin)},anchor=south},
                width=0.7*\textwidth, height=0.4*\textwidth,
                y=4cm,
                clip=false,
                %every tick label/.append style={font=\scriptsize}
            ]

        \draw[draw=none, left color=red!30, right color=red!10!white] (axis cs:\xint,0) rectangle (axis cs:\xmax,\ymax);

        \addplot[black!25,ultra thick,name path=D] {0.75-x/15};
        \addplot[black,ultra thick,name path=B] {1-sig(x,\C)};
        \addplot[black,ultra thick,dashed,name path=C] {1};

        \draw[ultra thick, green!50!black!50, arrows = {Stealth[inset=0pt, angle=60:6pt]-Stealth[inset=0pt, angle=60:6pt]}, shorten >= .5mm, shorten <= .5mm] (axis cs:\xopt,0.75-\xopt/15) -- (axis cs:\xopt,\yopt);
        %\draw[ultra thick, red!50] (axis cs:\xint,0) -- (axis cs:\xint,\ymax);

        \node[draw=none] at (axis cs:\xopt-1.2,0.835) {\scriptsize\color{green!50!black!50} \makecell{Best\\measurement}};
        \node[draw=none] at (axis cs:7.75,0.75) {\footnotesize Not economical};

        \end{axis}
    \end{tikzpicture}
    }
    \vspace*{-0.25cm}
    \caption{Illustrative \glsxtrshort{evii} vs. measurement cost curve} \label{fig:methodology-EVII-precision-cost-curve}
    \vspace*{0.15cm}
    \begin{singlespace}
      \raggedright
      \footnotesize{\it This figure shows a stylised relationship between the expected value of information (\glsxtrshort{evii}, black line) and the cost of measurement (grey line) as measurement precision varies (given by the variance of the likelihood fn along the x-axis).

      The `best measurement' (indicated is green) is the one which gives the highest net benefit (\glsxtrshort{evii} minus cost).}
    \end{singlespace}
\end{figure}